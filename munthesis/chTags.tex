%TCIDATA{Version=5.50.0.2960}
%TCIDATA{LaTeXparent=0,0,Master.tex}
                      
%TCIDATA{ChildDefaults=chapter:1,page:1}


\chapter{Tags and environments\label{chap:tags}}

In order to use tags, the Tag toolbar should be showing (View \TEXTsymbol{>}%
\TEXTsymbol{>}\ Toolbars). This toolbar has 3 \textquotedblleft
drop-down\textquotedblright\ menus.

\begin{itemize}
\item Item Tags (Alt-1)

\item Section/Body Tags (Alt-2)

\item Text\ Tags\ (Alt-3)
\end{itemize}

\section{Sections and Chapters\label{sec:sections}}

Chapters:\ No function key is assigned to chapter heads. Use the
Section/Body Tag drop-down (Alt-2).

F11 gives a new section head

\subsection{Subsections}

F12 gives a new subsection

\subsubsection{Subsubsections}

Shift-F11 gives a new subsubsection

\paragraph{Subsubsubsections}

Shift-F12 gives a new sub$^{3}$section (aka a paragraph). These are not
numbered and do not appear in the Table of\ Contents.

\section{Body tags}

The Section/Body tag drop-down (Alt-2) is used for environments that
typically do not nest and section headings. We've seen section headings
above. Some of the other environments available via Section/Body tags are

F3 Body text. This is the default kind of paragraph

\begin{center}
Centred. It was our yesterdayes resolution, and agreement, that we should to
day discourse the most distinctly, and particularly we could possible, of
the natural reasons, and their efficacy that have been hitherto alledged on
the one or other part, by the maintainers of the Positions, Aristotelian,
and Ptolomaique; and by the followers of the Copernican Systeme:
\end{center}

\begin{flushleft}
Flush left. And because Copernicus placing the Earth among the moveable
Bodies of Heaven, comes to constitute a Globe for the same like to a Planet;
it would be good that we began our disputation with the examination of what,
and how great the energy of the Peripateticks arguments is, when they
demonstrate, that this Hypothesis is impossible:
\end{flushleft}

\begin{flushright}
Flush right. Since that it is necessary to introduce in Nature, substances
different betwixt themselves, that is, the C\oe lestial, and Elementathat
impassible and immortal, this alterable and corruptible. Which argument
Aristotle handleth in his book De C\oe lo, insinuating it first, by some
discourses dependent on certain general assumptions, and afterwards
confirming it with experiments and perticular demonstrations: following the
same method, I will propound, and freely speak my judgement, submitting my
self to your censure, and particularly to Simplicius, a Stout Champion and
contender for the Aristotelian Doctrine.
\end{flushright}

\begin{quotation}
Long quotation. And the first Step of the Peripatetick arguments is that,
where Aristotle proveth the integrity and perfection of the World, telling
us, that it is not a simple line, nor a bare superficies, but a body adorned
with Longitude, Latitude, and Profundity.
\end{quotation}

\begin{quote}
Short Quote.
\end{quote}

Body and section tags should be applied only to top-level paragraphs.
Furthermore Item tags can not be used within Body tags. You can try to make
a list of quotations or a quotation that contains a list, but the
straight-forward approach is not likely to succeed. (These things can be
done with LaTeX; the limitation is with Scientific Word and Scientific
Workplace.)

\section{Item tags and environments that nest}

The \textquotedblleft Item Tag\textquotedblright\ drop-down (Alt-1) lists
all tags that nest. Some of these generate list-like environments in LaTeX.

\begin{itemize}
\item Bullet lists F8

\begin{enumerate}
\item Numbered lists F7

\begin{description}
\item[Description lists:] Use \textquotedblleft Description List
Item\textquotedblright\ on the Item Tag drop down.

\item[Depth:] You can nest item tags to a depth of 4, but no more

\begin{itemize}
\item This list item is at depth 4. LaTeX supports deeper nesting, but
Scientific Word and WorkPlace do not.
\end{itemize}
\end{description}

\item See also the \textquotedblleft code\textquotedblright\ and
\textquotedblleft indent\textquotedblright\ tags discussed in the next
chapter.
\end{enumerate}

\item F2 to step out a level.

\item 
%TCIMACRO{\TeXButton{B SingleSpaced}{\begin{singlespaced}}}%
%BeginExpansion
\begin{singlespaced}%
%EndExpansion
The \textquotedblleft singlespaced\textquotedblright\ environment is not
associated with a function key or tag. You can use TeX fields (Insert\ 
\TEXTsymbol{>}\TEXTsymbol{>}\ Typeset Object \TEXTsymbol{>}\TEXTsymbol{>}
TeX Field) to insert the required LaTeX magic. You can also use the
\textquotedblleft singlespaced\textquotedblright\ fragment.%
%TCIMACRO{\TeXButton{E SingleSpaced}{\end{singlespaced}}}%
%BeginExpansion
\end{singlespaced}%
%EndExpansion
\end{itemize}

\section{Theorems and proofs}

Theorem-like environments and proofs (perhaps surprisingly) nest. Use the
Item Tag drop-down (Alt-1) Be sure to use F2 (Remove Item Tag) after the end
of each.

\begin{axiom}[The axiom of equality]
All animals are equal.
\end{axiom}

\begin{claim}
That some animals are more equal than others.
\end{claim}

\begin{conclusion}
Whatever goes upon two legs is an enemy.
\end{conclusion}

\begin{condition}
No animal shall drink alcohol.
\end{condition}

\begin{corollary}
Whatever goes upon four legs, or has wings, is a friend.
\end{corollary}

\begin{conjecture}
All men are enemies. All animals are comrades.
\end{conjecture}

\begin{criterion}
No animal shall kill any other animal.
\end{criterion}

\begin{definition}[Insanity]
Doing the same thing over and over again while expecting different results.
\end{definition}

\begin{example}
Example is not the main thing in influencing others. It is the only thing.
\end{example}

\begin{exercise}
Find all errors in Cantor's diagonal proof that the continuum is larger than 
$\aleph _{0}$.
\end{exercise}

\begin{lemma}
What is yellow and equivalent to the axiom of choice. ... Zorn's Lemon.
\end{lemma}

\begin{notation}
We use $\left\langle E\right\rangle $ to represent that total function $f$
in $\Sigma \times \Sigma \rightarrow \mathbb{B}$ such that $f(\sigma ,\sigma
^{\prime })$ is equal to the value of the expression obtained by replacing,
in the boolean expression $E$, each unprimed identifier $i$ with $\sigma .i$
and each primed identifier $i^{\prime }$ with $\sigma ^{\prime }.i$.
\end{notation}

\begin{problem}
Show that $P=NP$ or that $P\neq NP$.
\end{problem}

\begin{proposition}
No animal shall wear clothes.
\end{proposition}

\begin{solution}
That which is not in the problem set is in the solution set.
\end{solution}

\begin{summary}
No question now, what had happened to the faces of the pigs. The creatures
outside looked from pig to man, and from man to pig, and from pig to man
again; but already it was impossible to say which was which.
\end{summary}

\begin{theorem}[The Curry/L\"{o}b paradox]
\label{thm:santa}I am Santa Claus.
\end{theorem}

\begin{remark}
Truth is an ill-defined concept.
\end{remark}

\begin{proof}
(Theorem \ref{thm:santa}.)

(a)\ Let $S$ be the statement \textquotedblleft If $S$ is true, then I am
Santa Claus\textquotedblright .

(b)\ Assume, for the moment that $S$ is true.

\begin{itemize}
\item Since $S$ is true and $S$ is the statement \textquotedblleft If $S$ is
true, then I am Santa Claus\textquotedblright , the statement
\textquotedblleft If $S$ is true, then I am Santa Claus\textquotedblright\
is true.

\item Therefore, if $S$ is true, then I am Santa Claus.

\item As we are assuming $S$ is true, I\ am (under that assumption)\ Santa
Claus.
\end{itemize}

(c) Since in (b), assuming $S$ to be true lead to the conclusion that I am
Santa Claus, we can see that \textquotedblleft If $S$ is true, then I am
Santa Claus\textquotedblright\ is true.

(d)\ By (c)\ and (a), $S$ is true.

(e)\ From (c), if $S$ is true, then I am Santa Claus.

(f)\ From (d) and (e), by modus ponens, I am Santa Claus.
\end{proof}

In Scientific Word and WorkPlace you can double click the tagged paragraph's
leader (i.e. the word(s) or symbols leading the paragraph) to enter an
optional \textquotedblleft custom label\textquotedblright ,\ as I\ did with
the Axiom, Definition, and Theorem above. This works for theorem-like
environments and description-list items. It does not work for proofs.

The numbering system that I have chosen is to number theorems sequentially
within each chapter and all other theorem-like entities as if they were also
theorems. You can use another numbering system if you want; to do so, edit
the preamble (Typeset \TEXTsymbol{>}\TEXTsymbol{>}\ Preamble) of the master
document.

\section{Fonts and sizes}

The Text Tag drop-down (Alt-3) is used for tags that operate at the
character level rather than the paragraph level.

\begin{itemize}
\item F4 to return to the default.

\item \textrm{Shift-F4 for roman}

\item \textbf{F5 for bold}

\item \textsf{Shift-F5 for sans-serif}

\item \emph{F6 for emphasized. Emphasized (by default) alternates between
italic and upright as it nests}

\item \textit{Shift-F6 for italic. Italic is always italic. Use italic for
multi-letter names in math mode.}

\item \texttt{F9 for typewriter}

\item $\mathcal{SHIFT}$-$\mathcal{F}9~\mathcal{FOR\ CALIGRAPHIC}$ Only
capitals and numerals. Best saved for math.

\item $\mathfrak{Frackur}$ Best saved for math

\item $\mathbb{BLACKBOARD\ BOLD}$ Only capitals. Use only for constant sets: 
$\mathbb{C}$ for the complex numbers, $\mathbb{R}$ for the real numbers, $%
\mathbb{Z}$ for the integers, $\mathbb{N}$ for the natural numbers, $\mathbb{%
B}$ for the Booleans.

\item \textsc{Small Capitals}.

\item Most font size changes needed are done automatically. You should never
or only very rarely use {\Huge size} {\scriptsize changing} {\LARGE text} 
{\tiny tags}.

\item \emph{\textbf{Combining text tags is possible, but rather tricky,
unless one tag is strictly within the other. Start by typing three letters,
say \textquotedblleft xyz\textquotedblright ; make all three emphasized.
Then make the y bold. Select the y and start typing. Finally delete the x
and the z.}}
\end{itemize}

$\mathit{\Gamma }\rho \varepsilon \varepsilon \kappa $ $\lambda \varepsilon
\tau \tau \varepsilon \rho \sigma $. Greek letters are available on the
Symbol Panels toolbar (View \TEXTsymbol{>}\TEXTsymbol{>}\ Toolbars...) and
from the keyboard. For example \textquotedblleft Control-g
p\textquotedblright\ gives $\pi $ and Control-g P gives $\Pi $. Missing
capital greek letters can be replaced by their Roman equivalents: $\mathrm{AB%
}\Gamma \Delta $. Likewise lowercase omicron is not available, but you can
use an italic oh, $o$. Some letters come in more than one variation:\ $%
\epsilon $ and $\varepsilon $, $\theta $ and $\vartheta $, $\rho $ and $%
\varrho $, $\pi $ and $\varpi $, $\phi $ and $\varphi $, $\sigma $ and $%
\varsigma $. I prefer $\varepsilon $ to $\epsilon $, as the latter is often
confused with $\in .$ (By the way, please do not use $\epsilon $ in place of 
$\in $.) Otherwise, I prefer $\theta $, $\rho $, $\pi $, and $\sigma $,
these being the more standard variants. Greek is not a text tag, so no
tricks are needed to combine Greek letters with text tags to make, for
example, bold Greek $\mathbf{AB\Gamma \Delta \alpha \beta \gamma \delta }$
or italic Greek $\mathit{AB\Gamma \Delta \alpha \beta \gamma \delta }$.
However most\ other text tags do not work with Greek letters; in particular,
there is no easy way to make upright lower-case greek letters.
