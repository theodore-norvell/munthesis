%TCIDATA{Version=5.50.0.2960}
%TCIDATA{LaTeXparent=0,0,Master.tex}
                      
%TCIDATA{ChildDefaults=chapter:1,page:1}


\chapter{Algorithms and listings}

\section{The code tag/environment}

For algorithms I use my own environment called \textquotedblleft
code\textquotedblright\ (Shift-F8). Before you use this you have to install
my code.sty file somewhere latex can find it. Here is an example

%TCIMACRO{\TeXButton{B SingleSpaced}{\begin{singlespaced}}}%
%BeginExpansion
\begin{singlespaced}%
%EndExpansion

\begin{code}
\textbf{let} $x\mid x\in\mathbb{N}\cdot$

\textbf{while} $x>0$ do

\begin{indent}
\item $x:=x-1$
\end{indent}
\end{code}

%TCIMACRO{\TeXButton{E SingleSpaced}{\end{singlespaced}}}%
%BeginExpansion
\end{singlespaced}%
%EndExpansion

Indentation is done with the indent tag (Shift-F7). To unindent use F2. Also
use F2 to get out of code. Please only use the indentation tag (Shift-F7)
inside of the code tag (Shift-F8). Unfortunately SWP limits the number of
indentations levels to 4 --- counting not indented at all. To get more
indentation, I\ start putting in 2-Em spaces (Insert \TEXTsymbol{>}%
\TEXTsymbol{>}\ Spacing \TEXTsymbol{>}\TEXTsymbol{>}\ Horizontal Space...).
Even the first line can be indented.

%TCIMACRO{\TeXButton{B SingleSpaced}{\begin{singlespaced}}}%
%BeginExpansion
\begin{singlespaced}%
%EndExpansion

\begin{code}
\begin{indent}
\item $\left\langle P\right\rangle $
\end{indent}

$\sqsubseteq $

\begin{indent}
\item \textbf{if} $B$ \textbf{then}

\begin{indent}
\item $\left\langle B\Rightarrow P\right\rangle $
\end{indent}

\item \textbf{else}

\begin{indent}
\item $\left\langle \lnot B\Rightarrow P\right\rangle $
\end{indent}
\end{indent}
\end{code}

%TCIMACRO{\TeXButton{E SingleSpaced}{\end{singlespaced}}}%
%BeginExpansion
\end{singlespaced}%
%EndExpansion

The fragment \textquotedblleft sscode\textquotedblright\ inserts a single
spaced code environment.

Make sure that all code belonging to a single example is in one code
environment. You should only see one leader (i.e., one green box containing
the word \textquotedblleft code\textquotedblright ). If there are several
leaders, delete all but the first using the back-space key. This often
needed after backing out of an indented section using F2. Figure~\ref%
{fig:extra-code} shows what can happen. If you are looking at this document
with SW or SWP, you will see the second leader in front of the word
\textquotedblleft assert\textquotedblright . If you are looking at the
typeset document, you will see extra space caused by having two code
environments where there should be one. After using F2 to get out of the
indent environment I\ should have used the backspace key to eliminate the
second green leader. You might wonder whether a similar trick is needed when
you use F2 to leave a nested indent environment; the answer is `no'.

%TCIMACRO{\TeXButton{B Figure}{\begin{figure}[tb]}}%
%BeginExpansion
\begin{figure}[tb]%
%EndExpansion
%TCIMACRO{\TeXButton{B SingleSpaced}{\begin{singlespaced}}}%
%BeginExpansion
\begin{singlespaced}%
%EndExpansion

\begin{code}
\textbf{assume} $t<\infty $

\textbf{var} $i:\in \mathbb{N}$

\textbf{while} $i>0$ \textbf{do}

\begin{indent}
\item $i:=i-1$
\end{indent}
\end{code}

\begin{code}
\textbf{assert} $t<\infty $
\end{code}

%TCIMACRO{\TeXButton{E SingleSpaced}{\end{singlespaced}}}%
%BeginExpansion
\end{singlespaced}%
%EndExpansion
%TCIMACRO{%
%\TeXButton{E Figure}{\caption{Incorrect spacing casued by an extra code environment}\label{fig:extra-code}
%\end{figure}}}%
%BeginExpansion
\caption{Incorrect spacing casued by an extra code environment}\label{fig:extra-code}
\end{figure}%
%EndExpansion

The fragment \textquotedblleft parcode\textquotedblright\ inserts code
consisting of two parallel threads. (It is not singlespaced, but you can use
the \textquotedblleft singlespaced\textquotedblright\ fragment to make it
so.) You may need to fiddle with the widths of the minipages.\ See Algorithm~%
\ref{alg:prod-cons} for an example.

%TCIMACRO{\TeXButton{B Algorithm}{\begin{algorithm}}}%
%BeginExpansion
\begin{algorithm}%
%EndExpansion
%TCIMACRO{\TeXButton{B centering}{\begin{centering}}}%
%BeginExpansion
\begin{centering}%
%EndExpansion
%TCIMACRO{\TeXButton{B tabular}{\begin{tabular}{r||l}}}%
%BeginExpansion
\begin{tabular}{r||l}%
%EndExpansion
%TCIMACRO{\TeXButton{B minipage}{\begin{minipage}[t]{2.25in}}}%
%BeginExpansion
\begin{minipage}[t]{2.25in}%
%EndExpansion
%TCIMACRO{\TeXButton{B SingleSpaced}{\begin{singlespaced}}}%
%BeginExpansion
\begin{singlespaced}%
%EndExpansion

\begin{code}
\textbf{while} $\mathit{true}$ \textbf{do}

\begin{indent}
\item compute next value

\item $P(e)\qquad $

\item fill the buffer

\item $V(f)$
\end{indent}
\end{code}

%TCIMACRO{\TeXButton{E SingleSpaced}{\end{singlespaced}}}%
%BeginExpansion
\end{singlespaced}%
%EndExpansion
%TCIMACRO{\TeXButton{E minipage}{\end{minipage}}}%
%BeginExpansion
\end{minipage}%
%EndExpansion
%TCIMACRO{\TeXButton{newcolumn}{&}}%
%BeginExpansion
&%
%EndExpansion
%TCIMACRO{\TeXButton{B minipage}{\begin{minipage}[t]{2.25in}}}%
%BeginExpansion
\begin{minipage}[t]{2.25in}%
%EndExpansion
%TCIMACRO{\TeXButton{B SingleSpaced}{\begin{singlespaced}}}%
%BeginExpansion
\begin{singlespaced}%
%EndExpansion

\begin{code}
\textbf{while} $\mathit{true}$ \textbf{do}

\begin{indent}
\item $P(f)$

\item empty the buffer

\item $V(e)$

\item use the value
\end{indent}
\end{code}

%TCIMACRO{\TeXButton{E SingleSpaced}{\end{singlespaced}}}%
%BeginExpansion
\end{singlespaced}%
%EndExpansion
%TCIMACRO{\TeXButton{E minipage}{\end{minipage}}}%
%BeginExpansion
\end{minipage}%
%EndExpansion
%TCIMACRO{\TeXButton{E tabular}{\end{tabular}}}%
%BeginExpansion
\end{tabular}%
%EndExpansion
%TCIMACRO{\TeXButton{E centering}{\end{centering}}}%
%BeginExpansion
\end{centering}%
%EndExpansion

%TCIMACRO{\TeXButton{Caption}{\caption{A producer and a consumer}}}%
%BeginExpansion
\caption{A producer and a consumer}%
%EndExpansion
\label{alg:prod-cons}%
%TCIMACRO{\TeXButton{E Algorithm}{\end{algorithm}}}%
%BeginExpansion
\end{algorithm}%
%EndExpansion

Sometimes you want algorithms to float like figures and tables. You can use
the \textquotedblleft figure\textquotedblright\ fragment as in Figure~\ref%
{fig:extra-code}. You can also use the \textquotedblleft
algorithm\textquotedblright\ fragment. Use Alt-4 and select
\textquotedblleft algorithm\textquotedblright . Edit the caption and marker;
add your code. Algorithms inserted using this fragment are singlespaced.
Algorithm~\ref{alg:subsetcons} is an example.

%TCIMACRO{\TeXButton{B Algorithm}{\begin{algorithm}\begin{singlespaced}}}%
%BeginExpansion
\begin{algorithm}\begin{singlespaced}%
%EndExpansion

\begin{code}
$\dot{q}_{\mathrm{start}}:=\epsilon$-closure$(q_{\mathrm{start}})\;;$

\textbf{var} $W:=\{\dot{q}_{\mathrm{start}}\}\cdot$

$\dot{Q}:=\emptyset\;;$

$\dot{T}:=\emptyset\;;$

\textbf{while} $W\neq\emptyset$ \textbf{do }$\mathbf{(}$

\begin{indent}
\item \textbf{let} $\dot{q}\mid\dot{q}\in W\cdot$

\item $W:=W-\{\dot{q}\}\;;$

\item $\dot{Q}:=\dot{Q}\cup\{\dot{q}\}\;;$

\item \textbf{for} \textbf{each} $a\in S\;$\textbf{do}$\;($

\begin{indent}
\item \textbf{let} $\dot{r}=\epsilon$-closure$(\delta(\dot{q},a))\cdot$

\item \textbf{if} $\dot{r}\neq\emptyset$ \textbf{then }$($

\begin{indent}
\item $\dot{T}:=\dot{T}\cup\{(\dot{q},$\textsf{\textquotedblleft}$a$\textsf{%
\textquotedblright}$,\dot{r})\}\;;$

\item \textbf{if} $\dot{r}\not \in \dot{Q}$ \textbf{then} $W:=W\cup\{\dot {r}%
\}$ \textbf{else skip}$\;)$
\end{indent}

\item \textbf{else skip} $)\;)$
\end{indent}
\end{indent}

$\dot{F}:=\{\dot{q}\in\dot{Q}\mid\dot{q}\cap F\neq\emptyset\}$
\end{code}

%TCIMACRO{\TeXButton{Caption}{\caption{The subset construction algorithm}}}%
%BeginExpansion
\caption{The subset construction algorithm}%
%EndExpansion
\label{alg:subsetcons}%
%TCIMACRO{\TeXButton{E Algorithm}{\end{singlespaced}\end{algorithm}}}%
%BeginExpansion
\end{singlespaced}\end{algorithm}%
%EndExpansion

\section{Listings}

Listings are lines of source code. If you want to include source code in
your thesis, you will want listings

\subsection{If you do want to use listings\label{sec:doWantListings}}

You must install the listings package somewhere latex can find it. You
should read the documentation: listings.pdf.

You will also want to install the fragments mentioned below.

You can edit the default settings by editing the \TEXTsymbol{\backslash}%
lstset command in the preamble (Typeset \TEXTsymbol{>}\TEXTsymbol{>}\
Preamble). For example, if all (or most of) the listing in your thesis are
in Java, add \textquotedblleft language=Java\textquotedblright\ to the list
of key/value pairs in the \TEXTsymbol{\backslash}lstset command in the
preamble of the Master document.

\subsection{In-situ listings}

In-situ listings are fragments of code that are placed right in your TeX
file. Typically you copy text from a source file and paste it in to
Scientific Workplace.

Here is a listing that appears \textquotedblleft displayed\textquotedblright
, i.e. as a paragraph in your document.

%TCIMACRO{\TeXButton{B SingleSpaced}{\begin{singlespaced}}}%
%BeginExpansion
\begin{singlespaced}%
%EndExpansion

%TCIMACRO{%
%\TeXButton{In-situ-listing}{\begin{lstlisting}[language=Pascal]
%program Hello ;
%begin
%   write("Hello");
%end.
%\end{lstlisting}}}%
%BeginExpansion
\begin{lstlisting}[language=Pascal]
program Hello ;
begin
   write("Hello");
end.
\end{lstlisting}%
%EndExpansion

%TCIMACRO{\TeXButton{E SingleSpaced}{\end{singlespaced}}}%
%BeginExpansion
\end{singlespaced}%
%EndExpansion

You can also have listings as floating elements, like figures and tables.
These have captions and labels, which means you can refer to them using
cross-references. See Listing \ref{lst:inSitu} for an example.

%TCIMACRO{\TeXButton{B SingleSpaced}{\begin{singlespaced}}}%
%BeginExpansion
\begin{singlespaced}%
%EndExpansion

%TCIMACRO{%
%\TeXButton{Floating-in-situ-listing}{\begin{lstlisting}[float=tb,language=Haskell,caption={This listing is {\it in situ}}, label=lst:inSitu]
%data Tree a = Empty | Branch a (Tree a) (Tree a)
%
%Flatten t = FlattenAndCat t []
%FlattenAndCat Empty xs = xs
%FlattenAndCat (Branch x left right ) ys
%   =  FlattenAndCat left (x :: FlattenAndCat right ys)
%
%five = let 2+2 = 5 in 2+2 
%
%\end{lstlisting}}}%
%BeginExpansion
\begin{lstlisting}[float=tb,language=Haskell,caption={This listing is {\it in situ}}, label=lst:inSitu]
data Tree a = Empty | Branch a (Tree a) (Tree a)

Flatten t = FlattenAndCat t []
FlattenAndCat Empty xs = xs
FlattenAndCat (Branch x left right ) ys
   =  FlattenAndCat left (x :: FlattenAndCat right ys)

five = let 2+2 = 5 in 2+2 

\end{lstlisting}%
%EndExpansion

%TCIMACRO{\TeXButton{E SingleSpaced}{\end{singlespaced}}}%
%BeginExpansion
\end{singlespaced}%
%EndExpansion

Use fragment \textquotedblleft listing (nonfloating)\textquotedblright\ for
nonfloating listings and \textquotedblleft listing
(floating)\textquotedblright\ for floating listings.

\subsection{Turning files into listings}

The listings package allows LaTeX to read in a source file and turn it into
a nicely typeset listing.

An example can be found in Listing \ref{lst:fromFileExample}.

%TCIMACRO{\TeXButton{B SingleSpaced}{\begin{singlespaced}}}%
%BeginExpansion
\begin{singlespaced}%
%EndExpansion

%TCIMACRO{%
%\TeXButton{Include of CommonParserHelper}{\lstinputlisting
%[float=tb,language=Java,caption={This listing comes from a file}, label=lst:fromFileExample,firstline=18]{code/CommonParserHelper.java}}}%
%BeginExpansion
\lstinputlisting
[float=tb,language=Java,caption={This listing comes from a file}, label=lst:fromFileExample,firstline=18]{code/CommonParserHelper.java}%
%EndExpansion

%TCIMACRO{\TeXButton{E SingleSpaced}{\end{singlespaced}}}%
%BeginExpansion
\end{singlespaced}%
%EndExpansion

Use the fragment \textquotedblleft listing (from file)\textquotedblright\
and edit the contents of the grey box that gets inserted.

By default these listings will float. If you don't want that, just remove
the \textquotedblleft float=tb\textquotedblright\ key/value pair. Also it is
best to remove the caption and label for nonfloating listings.

Note that floating listings can not exceed a page, which in a MUN\ thesis is
about 40 lines single spaced. In the example, I started this listing at line
18 of the file (firstline=18) so as to get the float to fit on a page. (The
first 17 lines are comments; and who wants to read those?)

\subsection{In-line listings}

In-line listings are short pieces of text that can be included in the normal
flow of a paragraph. Use the fragment \textquotedblleft listing
(inline)\textquotedblright . Here is an example:\ 
%TCIMACRO{%
%\TeXButton{Inline listing}{\lstinline[prebreak={},postbreak={}]@thesis.MagicAlgorithm.runSpell( i );@}}%
%BeginExpansion
\lstinline[prebreak={},postbreak={}]@thesis.MagicAlgorithm.runSpell( i );@%
%EndExpansion

\subsection{Typeset the master document}

If you use inputted file listings, only typeset the \textquotedblleft
Master\textquotedblright\ document. Typesetting a chapter on its own won't
work. The reason is that SW/SWP usually will move your document to a
temporary location for typesetting. But as it doesn't move your code files,
they can't be found from the temporary location. An exception is when SW/SWP
typesets a \textquotedblleft Master\textquotedblright\ document; then it
doesn't use a temporary location and sanity is restored.

\subsection{If you don't want to use listings}

With Typeset\ \TEXTsymbol{>}\TEXTsymbol{>}\ Options and Packages... remove
\textquotedblleft listings\textquotedblright\ from the package list. Then,
in the preamble (Typeset \TEXTsymbol{>}\TEXTsymbol{>}\ Preamble),\ remove
all commands that have to do with listings. You should also remove the List
of Listings from the frontMatter.tex document. This will improve compile
times and save you from having to install the \textquotedblleft
listings\textquotedblright\ package.
